\documentclass[11pt]{article}
\usepackage[hmargin=1in,vmargin=1in]{geometry}
\usepackage{xcolor}
\usepackage{amsmath,amssymb,amsfonts,url,sectsty,framed,tcolorbox,framed}
\usepackage{multicol}
\setlength{\columnsep}{0.5cm}
\usepackage{blindtext}
\usepackage{tikz}

\newcommand{\pf}{{\bf Proof: }}
\newtheorem{theorem}{Theorem}
\newtheorem{lemma}{Lemma}
\newtheorem{proposition}{Proposition}
\newtheorem{definition}{Definition}
\newtheorem{remark}{Remark}
\newcommand{\qed}{\hfill \rule{2mm}{2mm}}


\begin{document}
%%%%%%%%%%%%%%%%%%%%%%%%%%%%%%%%%%%%%%%%%%%%%%%%%%%%%%%%%%%%%%%%%%%%%

\noindent
\rule{\textwidth}{1pt}
\begin{center}
{\bf [CS304] Introduction to Cryptography and Network Security}
\end{center}
Course Instructor: Dr. Dibyendu Roy \hfill Winter 2022-2023\\
Scribed by: Rathva Kaushikkumar Sanjaybhai (202051156) \hfill Week : 1 (1st lecture\#)
\\
\rule{\textwidth}{1pt}

%%%%%%%%%%%%%%%%%%%%%%%%%%%%%%%%%%%%%%%%%%%%%%%%%%%%%%%%%%%
%write here

\section[short]{Cryptology:}
\subsection{Parts of Cryptology:}

\begin{definition}
\underline{{Cryptography}} : 
We develop/design algorith to make system secure.  
\end{definition}

\begin{definition}
    \underline{{Cryptoanalysis}} : We try to penetrate the security of system.

\end{definition}

\begin{definition}
    {Cryptology = Cryptography + Cryptoanalysis.}    
\end{definition}

\begin{remark}
    {NIST}(National Institute of Standards and Technology) 
    is a Institution that Standardizes Cryptographic Algorithms.    
\end{remark}    

\section{Encription and Decryption}
\begin{multicols}{2}

    \subsection{Encription}
    
    Encription can be defined by {\underbar{\(E(P,k)=C\)}}.\newline
    Encription is process to convert/transform plain
    (readable\footnote[2]{it's meaning is known by reading it and can be used directly where it's intended.})
    text into
    cipher(unreadable
    \footnote[3]{it's meaning can't known by reading it and can't used as intended directly.})
     text.
    
    \subsection{Decryption}
    Decryption can be defined by {\underbar{\(D(C,k)=P\)}}. \newline
    Decryption is process to convert/transform cipher text to plain text.
    \newline\newline
    Where, \newline
    $P = Plain text$ \newline
    $C = Cipher text$ \newline
    $k = Secret key$ \newline

    \columnbreak

    \subsection{Example:}

    ATM 1 $\rightarrow$ PIN 1 + X = Y1  \newline
    ATM 2 $\rightarrow$ PIN 2 + X = Y2  \newline
    ATM 3 $\rightarrow$ PIN 3 + X = Y3  \newline
    . \newline
    . \newline
    . \newline
    ATM 10 $\rightarrow$ PIN 10 + X = Y10 \newline

    \begin{remark}
        Here,$X \rightarrow Secret$
    \end{remark}
\end{multicols}

\newpage


\section{Cryptography :}
\begin {multicols}{2}
\subsection{Symetric key cryptograpy }
Both Encryption and Decryption keys are the same in this type of cryptography.
\newline
Encryption : $E(P,k)=C$<\newline
Decrytion : $D(C,k)=P$ \newline

Where , \newline
$P = plain text,$

\columnbreak
\subsection{Public key cryptograpy}
Encryption and Decryption keys are diffrent but both are related.\newline

There are two keys :
\begin{enumerate}
    \item Public key : which can be seen by anyone.
    \item Secret key : This key is kept secret and known reciever.This key is related to public key.
\end{enumerate}


\end{multicols}\\\\

\section{Cryptography provides following security services :-}
\begin{multicols}{2}
\subsection{Confidentiality(Secrecy) : }
\paragraph{}{It means that the massage is only known or understood by desired people.}

    \subsubsection{Plain text :}original massage.
    \subsubsection{Encription Algorithms :} function
    \subsubsection{Decryption Algorithm :} function
    \subsubsection{Cipher text :} un-readable form of plain text.
    \subsubsection{Encription key :} key
    \subsubsection{Decryption key :} key

\columnbreak

\subsection{Integrity}
\paragraph{}{Integrity means Text on both \underbar{Sender} and the 
    \underbar{Receiver} end is same.}
\subsection{Authentication}
\paragraph{}{Authentication is a process to identify desired person.}
\subsection{Non-repudiation}
\paragraph{}{A mechanism to prove that sender sent the message.}
\end{multicols}

\newpage

\section{CAESAR cipher :}
This cipher is named after \underline{Julius caesar}.It works by shifting letters of mases by an agreed number.\\
 Here we are taking agreed number =3.\\\\
 If we give/map all alphabet a number staring from 0.\\
 A $\rightarrow$ 0,
 B $\rightarrow$ 1,
 C $\rightarrow$ 2,
 .
 .
 .,
 Z $\rightarrow$ 25\\
 \\
 while Encrypting shift right all the letters by 3.\\\\
 plain text  $\rightarrow$ INTERNET\\
 agreed number = 3 $\rightarrow$ secret key \\
 Cipher text $\rightarrow$ LQWHUQHW\\\\
 while Decrypting shift left all the letters by 3.\\\\
Plain text $\rightarrow$ INTERNET\\
\end{document}

