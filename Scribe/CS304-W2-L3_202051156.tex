\documentclass[11pt]{article}
\usepackage[hmargin=1in,vmargin=1in]{geometry}
\usepackage{xcolor}
\usepackage{amsmath,amssymb,amsfonts,url,sectsty,framed,tcolorbox,framed}
\usepackage{multicol}
\setlength{\columnsep}{0.5cm}
\usepackage{blindtext}
\usepackage{tikz}

\newcommand{\pf}{{\bf Proof: }}
\newtheorem{theorem}{Theorem}
\newtheorem{lemma}{Lemma}
\newtheorem{proposition}{Proposition}
\newtheorem{definition}{Definition}
\newtheorem{remark}{Remark}
\newcommand{\qed}{\hfill \rule{2mm}{2mm}}


\begin{document}
%%%%%%%%%%%%%%%%%%%%%%%%%%%%%%%%%%%%%%%%%%%%%%%%%%%%%%%%%%%%%%%%%%%%%

\noindent
\rule{\textwidth}{1pt}
\begin{center}
{\bf [CS304] Introduction to Cryptography and Network Security}
\end{center}
Course Instructor: Dr. Dibyendu Roy \hfill Winter 2022-2023\\
Scribed by: Rathva Kaushikkumar Sanjaybhai (202051156) \hfill Week : 2 (3nd lecture \#)
\\
\rule{\textwidth}{1pt}

%%%%%%%%%%%%%%%%%%%%%%%%%%%%%%%%%%%%%%%%%%%%%%%%%%%%%%%%%%%
%write here
\section{Hill Cipher}
It's an cipher which has a nxn matrix as a key.\\

$a_{ij}\ \epsilon \  Z_{26} $\\\\
$A = (a_{ij})_{n\times n}$ \\\\
where , $ A\ is\ Secret key\ and\ has\ to\ be\ invertible.$\\\\
$M = \{m_1\ m_2\ .\ .\ .\ m_n\}\ \leftarrow \ {Z_{26}}^n\ possible strings. $\\\\
Here, M is plain text.
\subsection{Encryption :}
Cipher text $\ C\ =\ A \cdot M\ =\ (c_1\ c_2\ .\ .\ .\ c_n)$
\subsection{Decryption :}
Decrypted text $\ M\ =\ A^{-1}\cdot C$
\section{Substitution Box}
$S\ :\ \{A,B,.\ .\ .\,Z\}\ \rightarrow \ \{A,B,\ .\ .\ .\ ,Z\}$ \\\\
substitution box is Mapping on it self.\\
Where , $ P\ \rightarrow \ C\ =\ S(P) \ $\\\\
Here everything is known just Mapping/Sbox is kept secret.\\\\
If, Mapping is bijective their can be 26! such mapppings.\\
But if it's not then their are $\ 26^{26}\ $ mappings.\\\\
Secret key of the Sbox can be found out using brute force/Exhaustive search.
%%%%%%%%%%%%%%%%%%%%%%%%%%%%%%%%%%%%%%%%%%%%%%%%%%%%%%%%%%%%%%%%%%%%%
\end{document}
