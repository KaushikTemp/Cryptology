\documentclass[11pt]{article}
\usepackage[hmargin=1in,vmargin=1in]{geometry}
\usepackage{xcolor}
\usepackage{amsmath,amssymb,amsfonts,url,sectsty,framed,tcolorbox,framed}
\usepackage{multicol}
\setlength{\columnsep}{0.5cm}
\usepackage{blindtext}
\usepackage{tikz}

\newcommand{\pf}{{\bf Proof: }}
\newtheorem{theorem}{Theorem}
\newtheorem{lemma}{Lemma}
\newtheorem{proposition}{Proposition}
\newtheorem{definition}{Definition}
\newtheorem{remark}{Remark}
\newcommand{\qed}{\hfill \rule{2mm}{2mm}}


\begin{document}
%%%%%%%%%%%%%%%%%%%%%%%%%%%%%%%%%%%%%%%%%%%%%%%%%%%%%%%%%%%%%%%%%%%%%

\noindent
\rule{\textwidth}{1pt}
\begin{center}
{\bf [CS304] Introduction to Cryptography and Network Security}
\end{center}
Course Instructor: Dr. Dibyendu Roy \hfill Winter 2022-2023\\
Scribed by: Rathva Kaushikkumar Sanjaybhai (202051156) \hfill Week : 1 (2nd lecture \#)
\\
\rule{\textwidth}{1pt}

%%%%%%%%%%%%%%%%%%%%%%%%%%%%%%%%%%%%%%%%%%%%%%%%%%%%%%%%%%%
%write here
\section{One way function}
Using one way funtion we can easily findout the output from input.
But doing the otherway arround is computationaly difficult task.

\paragraph{Example : }For given two large prime numbers multiplication is a oneway function.
We can easily calculate answer to multiplication of amy two numbers in polinomial time.
But calculting back the prime numbers from the ouput.We have to facturize it in prime numbers to do that and it's computationally intensive task.

\section{Substitution box : }
 It's a function from $ A \rightarrow B $
where $|B| \le |A|$ \newline

example : \newline
$S1 : \{1,2,3,4\} \rightarrow \{1,2,3\}$ \newline

$S1(1) = 1 \  ,\   
S1(2) = 3 \  , \  
S1(3) = 2 \  , \  
S1(1) = 4$

\section{Transposition Cipher :}
This is a function which is mapped from domain to co-domain.\\\\
\textbf{Example :} $M = m_1 \  m_2 \ m_3 \  .\  .\  .\  m_t $ \\
 Here, M is plain text. \\
e : permutation on t elements $\rightarrow{}$ Secret\  key. \\\\
Encryption :\ $C = m_{e(1)} \ m_{e(2)} \ m_{e(3)} \ . \ . \ . m_{e(t)} $ \\
Here, C is cipher text.\\\\
Decryption :\ $n = C_{e^{-1}(1)} \ C_{e^{-1}(2)} \ C_{e^{-1}(3)} \ .\ .\ .\ C_{e^{-1}(t)}$\\
Here, n is the plain text that is decrypted from cipher text.
\newpage

\section{Permutation :}
\textbf{Example} : \\
C A E S A R = $m_1\ m_2\ m_3\ .\ .\ .\ m_6$ \\\\

$e\ : \{ \begin{array}{cccccc}
     1 & 2 & 3 & 4 & 5 & 6 \\
     6 & 4 & 1 & 3 & 5 & 2
\end{array}
\}_{Encryption}$
\\\\
Cipher text = R S C E A A = \ $C_1\ C_2\ C_3\ .\ .\ .\ C_6$ \\

$d = e^{-1}$\ :\  \{$\begin{array}{cccccc}
     1 & 2 & 3 & 4 & 5 & 6 \\
     3 & 6 & 4 & 2 & 5 & 1
\end{array}\}_{Decryption}$
\\\\
Plain text : C A E S A R

\section{Substitution Cipher : }
M =\ $m_1 \  m_2 \ m_3 \  .\  .\  .\  m_t $\\
A = \(a,b,c,d,...,z\)\ $M_i$\ $\epsilon$\ A \\\\
e : substitution from A to A.\\
e $\xrightarrow{}$ secret key. \\\\
ex. e(a) = z , e(b) = d , e(c) = a \\
a b c $\xrightarrow{}$ plain text \\
z d a $\xrightarrow{}$ cipher text

\begin{multicols}{2}
[\section{Affine cipher :}]
$A\ \rightarrow \ 0$ \\
$B\ \rightarrow \ 1$ \\
$C\ \rightarrow \ 2$ \\
.\\
.\\
.\\
$Z\ \rightarrow \ 25$ \\

\columnbreak
\subsection{Encryption function:}
$C\ =\ e(x,k)\ =\ (a \* x+b)mod\ 26\ =c$\\
where,\ $a,b,c\ \epsilon\ Z_{26}$

\subsection{Decryption function:}
$X\ =\ d(c,k)\ =\ ((c - b)a^{-1})mod\ 26\ =c$\\
\end{multicols}
\newpage
\subsection{Exapmple :}
$Z_6 = {0,1,2,3,4,5}\ \ \ x,y \ \epsilon \ Z_6 $
\begin{multicols}{2}
$ + \ mod \ 6 \ :\ +_6 \rightarrow \ Z = (x+y)%6 $
\\\\

$\begin{array}{ccccccc}
+_6 | & 0 & 1 & 2 & 3 & 4 & 5 \\
\hline
0\ | & 0 & 1 & 2 & 3 & 4 & 5 \\
1\ | & 1 & 2 & 3 & 4 & 5 & 0 \\
2\ | & 2 & 3 & 4 & 5 & 0 & 1 \\
3\ | & 3 & 4 & 5 & 0 & 1 & 2 \\
4\ | & 4 & 5 & 0 & 1 & 2 & 3 \\
5\ | & 5 & 0 & 1 & 2 & 3 & 4
\end{array}$

\\\\\\

$0\ne x \ \epsilon \ Z_6 $ \\
if $gcd(x,6)=1$ \\
then $\exist y $ such that \\
$x*_6y\ =\ 1$

\columnbreak

$ *_6 \  :\ (x*y)= Z%6 = y = x*_6y $ \\\\

$\begin{array}{lllllll}
*_6\ | & 0 & 1 & 2 & 3 & 4 & 5 \\
\hline
0\  |& 0 & 0 & 0 & 0 & 0 & 0 \\
1\  |& 0 & 1 & 2 & 3 & 4 & 5 \\
2\  |& 0 & 2 & 4 & 0 & 2 & 4 \\
3\  |& 0 & 3 & 0 & 3 & 0 & 3 \\
4\  |& 0 & 4 & 2 & 0 & 4 & 2 \\
5\  |& 0 & 5 & 4 & 3 & 2 & 1
\end{array}$

\\\\\\

$0\ne x \ \epsilon \ Z_m $ \\
if $gcd(x,m)=1$ \\
then $\exist y $ such that \\
$x*_my\ =\ 1$

\end{multicols}

for m and (xy - 1)\\
$xy\ -\ 1\ =\ tm$ \\
$1\ =\ t_1m\ +\ xy$ \ \ $t_1\ =\ (-t)$
\\\\\\
gcd(x,m) = ax + bm \\
\\\\
\textbf{Example : } m=7,x=3 \\\\
extended euclidion algo.\\\\
$ 1 = am+yz \\
1=3-(1X2)\\
=3+(-1X2)\\
=3+(-1)17+5X3\\
=6X3+(-1)X17\\\\
so, y=6,x=3,t=1,m=17.
$


\newpage


\section{Playfair cipher :}

secret key = playfair example \\
Here we are taking 5x5 matrix so we only have 25 distinct alphabets.So we are taking I=J.\\
Steps 
\begin{enumerate}
    \item Make a 5x5 matrix and add alphabets of secret key such that alphabets don't repeat.
       \\\\
       $\begin{array}{ccccc}
        P & L & A & Y & F \\
        I & R & E & X & M \\
        - & - & - & - & - \\
        - & - & - & - & - \\
        - & - & - & - & -
        \end{array}$
    \\
    \item Fill all the remaining alphabets in lexicographical order such that they don't repeat.
    \\\\
        $\begin{array}{ccccc}
        \hline
        |\ P & L & A & Y & F\ | \\
        |\ I & R & E & X & M\ | \\
        \hline
        B & C & D & G & H \\
        K & N & O & Q & S \\
        T & U & V & W & Z
        \end{array}$
    \\\\
    Here we have taken plain text HIDE.\\

    \item Break text into groups of 2 alphabets and if their are odd no of letters append z at the end.
    \item Now for the group mark both letters on the table and replace them with letters in same row relatively of the rectangle made by the group's letters.
    \item If both letters are in same row/column replace them with next element in circular meaner.\\

    For plain text HIDE \ $\rightarrow$ HI DE \\
    HI  DE\\
    $\ \ \downarrow\ \ \ $ \ $\downarrow$ \\
    BM  OD
\end{enumerate}
%%%%%%%%%%%%%%%%%%%%%%%%%%%%%%%%%%%%%%%%%%%%%%%%%%%%%%%%%%%%%%%%%%%%%
\end{document}
